\documentclass[11pt,a4paper]{article}

% Essential math packages
\usepackage{amsmath,amssymb,amsfonts,amsthm}

% Additional useful packages
\usepackage{graphicx}
\usepackage{physics}
\usepackage{mathtools}
\usepackage{enumitem}
\usepackage{booktabs}
\usepackage{xcolor}
\usepackage{hyperref}
\usepackage{microtype}

% Wide page with narrow margins
\usepackage[margin=1in]{geometry}

% Bibliography setup
\usepackage[
  backend=bibtex,
  style=numeric,
  sorting=none,
  maxbibnames=99
]{biblatex}
\addbibresource{jaynes-world.bib}

% Theorem environments
\newtheorem{theorem}{Theorem}
\newtheorem{lemma}[theorem]{Lemma}
\newtheorem{proposition}[theorem]{Proposition}
\newtheorem{corollary}[theorem]{Corollary}
\newtheorem{definition}{Definition}
\newtheorem{example}{Example}
\newtheorem{remark}{Remark}

\title{\LARGE \textbf{Jaynes World}}
\author{Neil D. Lawrence}
\date{March 7, 2025}

\begin{document}

\maketitle

\begin{abstract}
\end{abstract}

\section{Introduction}

In this paper we introduce Jaynes' world. Jaynes world is inspired by the work of John Conway in creating the \emph{Game of Life} and Stephen Wolfram in creating a series of one dimensional cellular automata. Like those worlds, the aim is to create a world where complex patterns can emerge from simple phenomena. 

\subsection{Gradient Descent}

We would like the world to be understandable by a typical high school physicist. We will therefore base the world on concepts that \emph{could} be understood at that level. But at its core will be one of the most mysterious concepts in science, entropy. 

Jaynes' world proceeds by following the trajectory of steepest ascent in entropy. The name of the world is inspired by Edwin Jaynes who was a great proponent of the principle ofmaximum entropy.

Like Conway's Game of Life and other cellular automata, our world will be defined on a grid. The grid consists of cells with unit length sides. The unit length gives the minimum distance that allows us to differentiate information in this universe. 

A fundamental parameter of Jaynes' world is given by the final maximum entropy configuration. In Jaynes' world we use Shannon's entropy which is defined in bits. We define the maximum entropy configuration to contain $N$ bits, where one bit of information is gained from knowing the outcome of a 50\% probability like a coin toss. This also leads to an information conservation law. The number of information carrying cells in the world remains fixed. Cells which don't contain information are said to be in a `vacuum' state.  

Conway used two dimensions to define his world, Wolfram focussed on one. Keeping dimensions low is one way of ensuring an aspect of simplicity in the world.

We go in the other direction to define simplicity, instead of specifying a number of dimensions we assume our world is non-parametric. From a practical sense this means there are as many dimensions as we need. For our world, that means we assume that there are $N$ dimensions. Because there is nothing we can do in $\infty$ dimensions that we can't do in $N$ dimensions.  The bits of information are stored in coordinates given by an $N$ dimensional vector $\mathbf{x}$ of integer values which specify cell locations in the grid.

We can represent the non-parametric nature of the world by introducing another $N$-dimensional vector $\boldsymbol{\theta}$, that define how many cells there are in each dimension. We define,
\[
\theta_i = \log_2 \ell_i,
\]
where $\ell_i$ is the width of the grid in the $i$th dimension. 

To add further complexity, we assume information can only change when it moves. In each `turn' of the game it can only move into a neigbouring cell.  

\subsection{First Observation in Jaynes' World}

Already with these definitions we can make some observations about the initial configuration of Jaynes' world. Regardless where we initialise to a sophisticated observer\footnote{We have not yet defined observers, but for the moment we assume they might exist.} will perceive that their world's configuration emerged from an origin point where entropy was minimised. This would be a point with o vacuum where all values of \mathbf{x} are coupled implying they are neighbours. 

Note that observers would perceive this initial state regardless of the actual initial configuration of their world, because when they reversed iterations they would be descending the entropy towards the origin point which would always be a minimum entropy configuration. 

This gives us a necessary initial condition of Jaynes' world, a state of minimum entropy and maximum entropic potential where all bits are neighbouring each other ensuring the entropy is zero.

The Jaynes' world game ends when the state of maximum entropy is reached. Maximum entropy distributions are characterised by a relationship between the Jacobian and the Hessian which comes to us from information geometry, the maximum entropy distribution is characterised by a Hessian which is the square of its Jacobian. 

The initial state of the world is at minimum entropy. Let's assume that the first gradient step defines the form of the probability distribution behind the entropy. That should maximise that form. In other words the first thing Jaynes' world does is give the intial minimal entropy configuration a probabilistic form through free-form optimisation of the variational distribution. This maximum entropy form is one where the Hessian is the square of the Jacobian.

\subsection{Dynamics}

The update rule for Jayne's world is `steepest gradient ascent'. At each iteration we take the move that gives us most entropy in the system.

However, as we'll now see, this is not the rule that gets us to the end of the game quickest. Indeed we might conjecture that it gives us the slowest route to the end of the game. 

Newton's method is an iterative method for finding roots of an equation. It can also be used to find the stationary points of a function. When applied in multiple dimensions it is called a second order method. When the method is applied, instead of ascending the gradient directly we change the parameters through a transformation of the Jacobian. At each iteration we weight the Jacobian by the inverse Hessian. This second order ascent method converges faster than standard gradient methods, but it can be prohibitive to implement because of the cost of computing and inverting the Hessian. Recall that in Jaynes' world, the form of the undelying probability distribution, $\rho(\cdot)$, is given by the first gradient step which ensures that the Hessian is the square of the Jacobian. In this case the \emph{optimal descent direction} becomes the \emph{Moore-Penrose} pseudo-inverse of the system's Jacobian. The Jaynes' world rule of steepest descent therefore goes in the inverse of the direction suggested by the optimal descent. In other words the Jaynes' world rule climbs by the most *suboptimal* route. The rule takes the scenic route.

Conceptually what happens is as follows, instead of ascending in the direction of the \emph{inverse Hessian}, we end up ascending in the direction provided by the \emph{Hessian}. As this proceeds it means that we ascend in flat bottomed valleys with very steep sides. Because the valley is flat bottomed, our algorithm continues along the base of the valley. This is what gives us the \emph{effective physics} in Jaynes' world. Locally it is represented by a low dimensional eigenstructure for the Hessian with most eigenvalues set to zero. 

However, as we proceed along these U-shaped valleys they eventually narrow as we approach a saddle point. At the saddle point one or more eigenvectors of the Hessian become negative and a branching occurs. The maximum entropy relationship between the Jacobian and Hessian implies a \emph{complex Jacobian} at this point, meaning our effective physics suddenly becomes much more complex. However, this rapidly resolves when we leave the saddle point and continue our ascent in another flat-bottomed valley. This ensures our physics has unpredictability despite its deterministic nature.

We might think of the complexity that Jaynes' world enables as being defined by the number of saddle points we pass through as we climb to the by our sub-optimal route. It may be possible to show using Morse theory that the Jaynes' world rule visits the largest number of saddle points on the way to the conclusion of the game and that number is $N$. 

\section{Summary}

I've introduced Jaynes' world, a conceptual world in the spirit of Wolfram's Rule 30 and Conway's Game of Life where `physics' is defined to progress in turns through simple rules, but complexity emerges through the interaction of those rules with an emergent information topography. I hope that Jaynes' world will provide a useful framework for engagement with more complex ideas in physics and their influence on the wider sciences and humanities.

\section*{Acknowledgements}

I would like to acknowledge the profound influence of David J. C. MacKay, whose approach to science exemplified clarity of thought, quantitative rigor, and an infectious enthusiasm. MacKay's work on information theory \cite{MacKay-information03} and sustainable energy \cite{MacKay-energy08} both demonstrate how principled mathematical thinking can illuminate complex systems across disciplines. His ability to communicate sophisticated ideas with elegant simplicity continues to inspire my work and that of so many colleagues. His motto `everything's connected' came to him through John Bridle and came to me from David. This motto is the foundation of Jaynes' world.

\end{document}
