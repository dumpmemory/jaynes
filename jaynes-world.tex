\documentclass[11pt,a4paper]{article}

% Essential math packages
\usepackage{amsmath,amssymb,amsfonts,amsthm}

% Additional useful packages
\usepackage{graphicx}
\usepackage{physics}
\usepackage{mathtools}
\usepackage{enumitem}
\usepackage{booktabs}
\usepackage{xcolor}
\usepackage{hyperref}
\usepackage{microtype}

% Wide page with narrow margins
\usepackage[margin=1in]{geometry}

% Bibliography setup
\usepackage[
  backend=bibtex,
  style=numeric,
  sorting=none,
  maxbibnames=99
]{biblatex}
\addbibresource{jaynes-world.bib}

% Theorem environments
\newtheorem{theorem}{Theorem}
\newtheorem{lemma}[theorem]{Lemma}
\newtheorem{proposition}[theorem]{Proposition}
\newtheorem{corollary}[theorem]{Corollary}
\newtheorem{definition}{Definition}
\newtheorem{example}{Example}
\newtheorem{remark}{Remark}

\title{\LARGE \textbf{Jaynes World}}
\author{Neil D. Lawrence}
\date{March 7, 2025}

\begin{document}

\maketitle

\begin{abstract}
\end{abstract}

\section{Introduction}

In this paper we introduce Jaynes' world. Jaynes world is inspired by the work of John Conway in creating the \emph{Game of Life} and Stephen Wolfram in creating a series of one dimensional cellular automata. Like those worlds, the aim is to create a world where complex patterns can emerge from simple phenomena. 

\subsection{Gradient Descent}

We would like the world to be understandable by a typical high school physicist. We will therefore base the world on concepts that \emph{could} be understood at that level. But at its core will be one of the most mysterious concepts in science, entropy. 

Jaynes' world proceeds by following the trajectory of steepest descent in an \emph{entropic potential}. The entropic potential is defined such that the steepest direction is the direction of maximum entropy production. The name of the world can be traced to this idea because Edwin Jaynes was a great prooponet of the principle of maximum entropy for deriving physical laws and fitting data driven models. 

Like Conway's Game of Life and other cellular automata, our world will be defined on a grid. The grid consists of cells with unit length sides. The unit length gives the minimum distance that allows us to differentiate information in this universe. 

Because we are defining an entropic potential, a fundamental parameter of our world is how many bits of information it holds. We will define the bits of information in the world to be $N$. That gives us an upper bound on the entropic potential, it's maximum is $N$ bits, where one bit of information is gained from knowing the outcome of a 50\% probability like a coin toss. There is an information conservation law, the number of information containing cells in the world remains fixed. Points which don't contain information are defined to be `vacuum'. 

Conway used two dimensions to define his world, Wolfram focussed on one. Keeping dimensions low is one way of ensuring an aspect of simplicity in the world.

We nake another direction to simplicity, instead of specifying dimensions we assume our world is non-parametric in dimensions so there are as many dimensions as there are bits in the world, and that means in the first instance we will assume the world is $N$-dimensional. These bits of information are represented in our world's configuration which is given by an $N$ dimensional vector $\mathbf{x}$ of integer cooridinate values on the grid.

The parameters of the world are given by another $N$-dimensional vector $\boldsymbol{\theta}$, they represent lengthscales through the following relationship,
\[
\theta_i = \log_2 \ell_i,
\]
where $\ell_i$ is the width of the grid in the $i$th dimension. 

Our final condition is that information can only move through the vaccum to a neighbouring cell in each iteraction. 

\subsection{First Observation in Jaynes' World}

Already with these definitions we can make some observations about the initial configuration of Jaynes' world. Regardless where we initialise to a sophisticated observer\footnote{We have not yet defined observers, but for the moment we assume they might exist.} will perceive that their world's configuration emerged from an origin point where entropy was minimised. This would be a point with o vacuum where all values of \mathbf{x} are coupled implying they are neighbours. 

Note that observers would perceive this regardless of the actuall initial configuration of their world. 

This gives us a necessary initial condition of Jaynes' world, a state of minimum entropy and maximum entropic potential where all bits are information connected. Having recovered a \emph{necessary} initial condition we can now look at the other boundary.

The final condition should be a point where the entropic potential is minimised, so the world is in a state of maximum entropy. This state is characterised by a relationship between the Jacobian and the Hessian which comes to us from information geometry, the maximum entropy distribution is characterised by a Hessian which is the square of its Jacobian. 

\subsection{Dynamics}

The dynamics of Jaynes' world come from gradient descent, but the relationship between Jacobian and Hessian inspires us to consider a variation on gradient descent that incorporates the Hessian.

Newton's method is an iterative method for finding roots of an equation, and can be modified to find the minima of a function. Instead of descending the gradient directly by changing the parameters through a scaled version of the Jacobian, we weight the Jacobian by the inverse Hessian. These second order descent methods converge faster than standard gradient methods, but are often prohibitive to implement because of the cost of inverting the Hessian. However, the condition on the dynamics that comes from our system's minima now comes to our aid.

If the Hessian is the square of the Jacobian, then the descent direction becomes the \emph{Moore-Penrose} pseudo-inverse of the systems Jacobian Jacobian. j.

\printbibliography

\end{document}
